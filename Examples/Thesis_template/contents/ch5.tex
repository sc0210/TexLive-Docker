\chapter{結論與未來工作}
\label{chapter:conclusion}
本研究將擅長解決複雜解空間問題的搜尋經濟學演算法應用於霧運算系統的伺服器配置上,解決原本需要考量成本、可連線裝置數量、伺服器工作處理量、延遲時間、覆蓋率及伺服器服務容量互相影響所形成的複雜問題。基於霧
運算系統階層式分散的架構改良配置演算法,結合搜尋經濟學演算法原本切割解空間的作法及霧運算系統配置受到閘道器配置數量影響的特性,發展出具有依據閘道器數量切割並衡量解空間潛力的伺服器配置演算法。在基於閘道
器數量所切割的解空間中,分配可全域移動的搜尋者及固定區域的商品達成全域的貪婪搜尋及區域的多樣搜尋。整體搜尋方向也可因應不同的資料集所產生的特性做策略上的調整,調整參數更細分為對目標值貪婪的商品數量、對
潛力區域貪婪的參賽者數量、對局部增加多樣性的區域數量以及全域多樣性的搜尋者數量。此種搜尋策略大幅度的提升配置演算法整體投入搜尋資源的效率,避免浪費搜尋資源在已搜尋過的區域或是局部沒有潛力的區域。利用區
域投資而得來的全域新解及被投資產生的區域新解計算區域期望值,有效的將資源利用在尚未開發或是開發中且曾經產出較佳目標值的區域。針對尚未收斂的區域,本研究新增的運算子(trade)不論是否有投入搜尋資源,將其
視為重點搜尋區域的做法也增加區域搜尋(local search)對於解空間探索的能力,透過調整$\gamma$值使...