\chapter{背景知識與相關研究}
\label{chapter:rtl-work}
\section{霧運算系統架構}
\label{section:architechture}


\section{問題定義}
\label{section:problem-definition}
%分配fog computing server會影響到的項目
本研究基於Lin \cite{lin_2018}針對智慧物流中心配置霧運算系統的問題定義衡量配置演算法。該研究基於上述霧運算架構特性,定義衡量霧運算架構的目標函式如下:
\begin{flalign}
    \label{equa:objective}
    &f = C_{l} + C_s + P,&
\end{flalign}
其中$C_{ij}$為衡量節點$i$及節點$j$的連線成本,$C_s$為所有伺服器的建置成本,$P$為懲罰成本。計算方式分別如下:
\begin{flalign}
    \label{equa:link-cost}
    &C_l = c_f \sum_{(i,j)\in\{s\}\times\Omega_G\cup\Omega_G\times\Omega_F\cup\Omega_F\times\Omega_E} x_{ij} \cdot d_{ij},&\\
    \label{equa:server-cost}
    &C_s = c_G \sum_{m\in\Omega_G}g_m + c_F \sum_{n\in\Omega_F} f_n + c_E \sum_{t\in\Omega_E} q_t,&\\
    \label{equa:penalty-cost}
    &P = \kappa \cdot (\eta_{\text{link}} + \eta_{\text{demand}} + \eta_{\text{latency}} + \eta_{\text{cover}} + \eta_{\text{capacity}}),&
\end{flalign}
公式 \ref{equa:link-cost}中,$c_f$為配置一單位光纖的價格,$x_{ij}$與$d_{ij}$分別為第$i$個節點與第$j$個節點間有無配置光纖的二進制數及其距離。若節點$i$、$j$間有連線需求,則
依照距離計算成本。公式 \ref{equa:server-cost}的$c_G$、$c_F$及$c_E$分別為安裝一個閘道器、霧伺服器及邊緣伺服器的成本。$g_m$、$f_n$及$q_t$則分別代表候選位置有無安裝閘道器、霧
伺服器及邊緣伺服器的二進制數。若有配置裝置則按照數量計算所有已配置伺服器的成本。在實際的霧運算系統中,有些限制是無法在解空間中定義並且從解空間中分離的。這些限制會以增加目標值
(objective value)的方式避免,有許多項目超出限制的霧運算系統將獲得極高的目標值,增加目標值的方式如公式 \ref{equa:penalty-cost}。$\kappa$為懲罰成本,
$\eta_{\text{link}}$、$\eta_{\text{demand}}$、$\eta_{\text{latency}}$、$\eta_{\text{cover}}$及$\eta_{\text{capacity}}$是在統計一個配置完的的霧運算系統有多少項目是超出限制
的。計算方式分別如下:
\begin{enumerate}[\hspace{2em}(\xCJKnumber{\arabic{enumi}})]
    \item 連線限制$\eta_{\text{link}}$:計算有多少的使用者或伺服器沒有被服務,即任意使用者、邊緣伺服器或霧伺服器$i$沒有分別被配置到邊緣伺服器、霧伺服器或閘道器$j$的情況是不被允許的:
    \begin{flalign}
        \label{equa:eta-link}
        &\eta_{\text{link}} \leftarrow \eta_{\text{link}}+1\text{,\quad if } x_{ij}=0,&
    \end{flalign}
    $x_{ij}$為代表裝置$i$及裝置$j$間有無配置線路的布林值。
    \item 服務限制$\eta_{\text{demand}}$:限制著每個伺服器能夠處理的服務量。每個使用者會有各自產生的基本服務需求量$r_A$,此服務需求量會由服務該使用者的邊緣伺服器接收。而
    霧伺服器及閘道器也將分別接收其服務的邊緣伺服器及霧伺服器的服務需求。邊緣伺服器$t$、霧伺服器$n$及閘道器$m$都有其能負荷的最大處理需求量$H^E_t$、$H^F_n$及$H^G_m$。對於任
    意伺服器或閘道器$j$,從低其一階的裝置$i$接收的服務需求量超出負荷是不被允許的:
    \begin{flalign}
        \label{equa:eta-demand}
        &\eta_{\text{demand}} \leftarrow \eta_{\text{demand}}+1\text{,\quad if } \sum_{k\in\Omega_i}r_i\cdot x_{jk} > H^E_j.&
    \end{flalign}
    \item 延遲限制$\eta_{\text{latency}}$:每個伺服器$i$的延遲時間是其所有服務的裝置$j$所造成延遲時間的總和,是由被服務裝置所產生的資料量$L_j$及其資料傳輸率$\gamma_j$而
    得。此項限制任意被配置伺服器$i$與其所有服務的裝置$j$的延遲率和大於延遲時間門檻$D_{ij}$:
    \begin{flalign}
        \label{equa:eta-latency}
        &\eta_{\text{latency}} \leftarrow \eta_{\text{latency}}+1 \text{,\quad if } \sum_{j}{L_j\over(x_{ij}\cdot\gamma_j)}>D_{ij}.&
    \end{flalign}
    \item 覆蓋限制$\eta_{\text{cover}}$:每個使用者附近至少要配置一個邊緣伺服器。對於任意使用者$k$,服務該使用者的邊緣伺服器$t$距離$d_{tk}$不得大於邊緣伺服器服務的半徑$R_E$:
    \begin{flalign}
        \label{equa:eta-cover}
        &\eta_{\text{cover}} \leftarrow \eta_{\text{cover}}+1 \text{,\quad if } x_{tk}\cdot d_{tk} > {R_E\over2}.&
    \end{flalign}
    \item 容量限制$\eta_{\text{capacity}}$:邊緣伺服器、霧伺服器及閘道器分別能容納的最大裝置容量分別為$N_E$、$N_F$及$N_G$。$\eta_{\text{capacity}}$計算任意超出裝置自身最
    大容量的伺服器$j$的數量:
    \begin{flalign}
        \label{equa:eta-capacity}
        &\eta_{\text{capacity}} \leftarrow \eta_{\text{capacity}}+1 \text{,\quad if } \sum_{k\in\Omega_i}x_{jk}\leq N_j.&
    \end{flalign}
\end{enumerate}

目標值將建置整個系統的成本、服務品質、延遲率及效能等考量在內,並藉由懲罰目標值避免伺服器超出負荷、斷線、高延遲等現象。例如圖 \ref{fig:objective}中左邊的範例,配置1個閘道器、1個霧伺服器、
2個邊緣伺服器及8單位的光纖。假設邊緣伺服器最多能夠服務3個使用者,編號(a)的邊緣伺服器服務5個使用者($\eta_{\text{capacity}} = 1$)、且未連線到任何霧伺服器($\eta_{\text{link}} = 1$)
,沒有違反其餘限制項目($\eta_{\text{demand}} = \eta_{\text{latency}} = \eta_{\text{cover}} = 0$),該配置的目標值為$c_G + c_F + 2c_E + 8c_f + 2\kappa$。若能夠將
配置改為右邊的範例,則目標值將降為$c_G + c_F + 2c_E + 9c_f$。在目標函數所形成的解空間中尋找到的最佳解將能以最低成本建置低延遲、高覆蓋率、在合理的伺服器負載下建置所有使用者
都能得到運算資源的霧運算系統。




\subsection{離散猴群基因演算法}
\label{section:dmga}




\section{總結}
...