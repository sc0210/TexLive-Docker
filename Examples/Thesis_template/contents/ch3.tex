\chapter{方法}
\label{chapter:method}
\section{符號說明}
\begin{table}[h]
    \small
    \centering
    % \caption{SEFSD與各演算法差異百分比}
    %     \newcommand{\xc}[1]{\multicolumn{1}{c}{#1}}
    %     \newcommand{\z}{\phantom{zzz}}
    %     \newcommand{\za}{\phantom{zz}}
    \begin{tabular}{lll}
        \centering
        符號        &   &代表意義     \\
        $k$         &   &區域的數量。 \\
        $r_i$       &   &第$i$個區域。 \\
        $R$         &   &區域的集合,$R=\{r_1, r_2, \dots, r_k\}$。 \\
        $m$         &   &搜尋者的數量。 \\
        $s^i_j$     &   &投資區域$i$的搜尋者$j$ \\
        $S$         &   &搜尋者的集合,$S=\{s_1, s_2, \dots, s_m\}$。 \\
        $n$         &   &區域內候選商品的數量。 \\
        $g^i_l$     &   &屬於第$i$個區域的第$l$個商品。 \\
        $G^i$       &   &屬於第$i$個區域的商品所形成的集合,$G^i=\{g^i_1, g^i_2, \dots, g^i_n\}$。 \\
        $t^a_i$     &   &區域$i$連續被投資的疊代數。 \\
        $t^b_i$     &   &區域$i$連續未被投資的疊代數。 \\
        $\mu_i$     &   &區域$i$連續被投資比率。 \\
        $\nu_i$     &   &投資區域$i$的投資者平均獲得的目標值。 \\
        $\rho_i$    &   &屬於第$i$個區域的商品平均獲得的目標值。 \\
        $E_i$       &   &區域$i$的期望值。 \\
        $E'_i$      &   &未收斂區域$i$調整後的期望值。 \\
        $o$         &   &參賽者數量。 \\
        $\gamma$    &   &貿易調整未收斂區域期望值的權重參數。 \\
        $t$         &   &疊代次數。 \\
        $d$         &   &擁有最佳目標值的解。 \\
    \end{tabular}
    \label{table:percentage}
    \end{table}
