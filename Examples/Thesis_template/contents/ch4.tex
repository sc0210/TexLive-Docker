\chapter{實驗結果與討論}
\label{chapter:exp}
\section{實驗環境及設定}
本研究於一台配有兩顆型號為Intel Xeon Silver 4110 (2.1GHz, 11 MB cache, 8 cores)的CPU、16 GB 記憶體及搭載作業系統 Ubuntu 18.04 LTS 的工作站進行實驗。實驗程式碼是以 C++ 撰
寫,編譯器為 gcc 7.4.0。實驗將比較本研究所提之基於搜尋經濟學演算法的霧運算系統配置方法與基於規則的最佳優先演算法(top first) \cite{li_2018}、早期的啟發式演算法——基因演算
法 \cite{holland_1962}及兩個近期的研究分別為基於離散蝙蝠演算法的工作配置演算法 \cite{mishra_2018}及解霧運算系統配置的離散猴群基因演算法 \cite{lin_2018}。基於規則的最佳優先演
算法在策略上最為貪婪,由於 NP-hard 問題無法與窮舉法做比較,故本研究選擇此類有規則限定的疊代演算法。最佳優先演算法優先分配負荷最重的伺服器,在理想情況下(即所有使用者都非
常集中且其附近都有數量足夠的伺服器候選配置地點),其能夠將霧運算系統配置最佳化。基因演算法則是早期最具代表性的超啟發式演算法,已有許多研究將其應用於各種不同性質的配置問題
中,且達到不錯的效果。較近期研究的離散蝙蝠演算法應用於工作配置問題所解決的問題解空間與本研究問題定義都屬於複雜解空間,此實驗將其改實作於霧運算系統配置問題上並比較本研究提
出方法與其在複雜解空間的成效。離散猴群基因演算法是解決霧運算系統配置問題的當前最新演算法,其研究中僅應用於一個空間大小$200m \times 180m$的物流中心,閘道器、霧伺服器、邊緣
伺服器候選位置數量及使用者數量分別為$3$、$15$、$70$及$500$個。本實驗將其應用在閘道器、霧伺服器、邊緣伺服器及使用者數量都更多的複雜資料集中,比較本研究所提配置法與原本即為
霧運算系統配置演算法的效果。