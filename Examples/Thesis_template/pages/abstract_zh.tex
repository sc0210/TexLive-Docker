\begin{abstractzh}
無線感測網路(Wireless Sensor Networks,簡稱 WSN)在現代科技中扮演著越來越重要的角色,但它們面臨著許多挑戰。其中一個主要的挑戰是如何有效地處理數據,因為傳輸過程中可能發生許多錯誤和丟包。解決這個問題的一種方法是利用元學習(Meta Learning)。元學習是一種學習如何學習的方法,它通常被應用於機器學習領域。在 WSN 中,元學習可以幫助系統學習如何在不同的網絡環境中自適應地處理數據。通過利用元學習的技術,WSN 可以根據網絡環境自動調整自己的參數和策略,提高整個系統的效率和準確性。具體來說,元學習可以幫助 WSN 實現以下目標:
在不同的網絡環境中自適應地調整數據處理策略,從而提高系統的效率和準確性。
自動調整系統的參數,從而適應不同的網絡場景和任務。學習如何優化網絡拓撲結構,從而提高網絡的穩定性和可靠性。
總之,元學習可以幫助 WSN 自動學習和優化系統的性能,從而解決數據處理方面的問題。這將有助於提高 WSN 的效率、可靠性和應用範圍,推動其在現代科技中的應用。

\providecommand{\keywords}[1]
{
\small  
\textbf{{關鍵字:}} #1
}
\keywords{自適應元學習、元學習、無線感測網路}
\end{abstractzh}